%%%%%%%%%%%%%%%%%%%%%%
% This is an example presentation made with Christopher Gandrud's unofficial LSE Beamer theme
% Updated 27 December 2011
%%%%%%%%%%%%%%%%%%%%%%

\documentclass{beamer}
\usetheme{LSE}
\usepackage{color}
\usepackage{hyperref}
	\hypersetup{
		colorlinks=true
		linkcolor=black
		}
\usepackage{graphics}
\usepackage{tikz}
\usepackage{booktabs}


%%%%%%%%%%%%%%%%%%%%%%%%%%%%%%%% Title Slide %%%%%%%%%%%%%%%%%%%%%%%%%%
\title[Getting What You Want]{Getting What You Want: Information \& Crisis Management in Ireland \& Korea}
\author[]{
    \href{mailto:c.gandrud@lse.ac.uk}{Christopher Gandrud} \and         \href{mailto:{m.a.okeeffee@lse.ac.uk}}{M\'{i}che\'{a}l O'Keeffe}
}
\date[Glope 2011]{Waseda University GLOPE II Conference, January 2011}


\begin{document}

\frame{\titlepage}

\section[Outline]{}
\frame{\tableofcontents}

%%%%%%%%%%%%%%%%%%%%%%%%%%%%%%%%%
\section{Introduction}
\frame{
    \frametitle{Motivation}
        \begin{itemize}
            \item<1-> Banking crises are relatively frequent (Reinhart \& Rogoff 2009) and almost always prompt a policy response (Rosas 2009).
            \item<2-> There are a wide variety of responses that can be chosen, including: {\emph{liability guarantees, recapitalisations, liquidity support, mergers, and nationalisations}}
            \item<3->{\bf{Questions:}}
                \begin{itemize}
                    \item<3-> Why do governments choose the banking crisis responses that they do?
                    \item<4-> Why are their choices often {\emph{not}} aligned with their preferences?
                \end{itemize} 
        \end{itemize}
}



\end{document}